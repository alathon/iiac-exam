\setcounter{section}{0}
\setcounter{figure}{0}
\setcounter{table}{0}
\newpage
\textbf{\LARGE{{Part two: Analysis, discussion and reflection}}}\\

\todo[inline]{Lead-in text for this part. Describe what the
section aims to provide. 1/2 to 1 page.}

Part two of this report explores the question of \textit{why}
the design process occured as it did. We attempt to provide a
partial answer to this question by considering
\begin{inparaenum}[1)]
\item what is a design process
\item a general description of techniques used in the design process
\item an overview of the entire design process from start til end
\item a detailed look at a few of the key choices and techniques employed.
\end{inparaenum}

The reader is expected to be familiar with details about \gomonkey{} 
from part one of the report.

\section{Participatory design in IT}
\todo[inline]{Describe what participatory design is, describe the
phases of design proposed by the course book and what each phase
represents and why each phase is important.}

IT projects vary widely in size, shape and form. From the potentially small, painless 
setup of Google Apps for a company of 3 employees, to the large-scale migration of
Jyske Bank's IT infrastructure after 2 years of planning\cite{jyskebank}.
A large portion of the work done in this report is based upon the general
practices and techniques proposed by \cite{bodker2004participatory}. In the book
they describe a search for techniques, concepts and ideologies which may apply
to the design and planning of IT projects on a general level. A specific method called 
\must{}\footnote{(a Danish acronym for theories and methods of initial analysis and 
design activities)} is proposed consisting of an
\begin{inparaenum}[1)]
\item initiation phase
\item in-line analysis phase
\item in-depth analysis phase and
\item innovation phase.
\end{inparaenum}

\cite{bodker2004participatory} define an IT design project as

\begin{figure*}[h!]
    \centering
    \textbf{IT design project}: A project conducted at a company that reveals
    goals, defines problems, and indicates solutions, with the aim of designing
    sustainable uses of IT based on a specific problem within the company.
\end{figure*}

Section~\ref{sec:process} describes how the authors realized an IT design
project using the \must{} method for \gomonkey{}. Readers are encouraged to read 
\cite{bodker2004participatory} for a more complete picture of how \must{} works 
as a general concept.

\section{Overview of techniques employed}
\todo[inline]{Describe a little about design techniques in general, cite
the coursebook and relevant articles. 3-6 sentences}

The techniques listed below were used in one or more of the design phases.
\todo[inline]{Below: Describe every technique used and how it was used, cite 
relevant articles for other uses or longer description}
\begin{description}
    \item [Baseline Planning]
    \item [Interview]
    \item [Observation]
    \item [Diagnostic Mapping]
    \item [Market study]
    \item [Think-aloud experiment]
    \item [Document analysis]
    \item [Developing scenarios]
    \item [Experimenting with prototypes]
    \item [Workshop]
    \item [Future workshop]
\end{description}

\section{From beginning to end: Our process of design} \label{sec:process}
\todo[inline]{Describe what we did from the very beginning
til the end. 2-3 pages}

\section{Combining the in-depth and innovation phases}
\todo[inline]{Describe how we combined the in-depth and
innovation phases, and how the literature also supports
this in certain cases. 2-3 pages}

\section{Ethnographically inspired analysis}
\todo[inline]{Describe how to combine techniques to provide
an ethnographically inspired analysis. It is not good enough
to know \textit{what} they do, we must also know \textit{why}.
1-2 pages}

\section{Observation as a technique}
\todo[inline]{Describe how observation works as a technique.
Describe how others have used observation to positive effect.
Describe how we have used observation, and what some of the
pitfalls are. Briefly cite and describe one alternative to
observation. 1-2 pages}

\section{User stories as a technique}
\todo[inline]{Describe how user stories work as a technique
to provide overview and structure workflow. Describe how we
performed this and provide relevant sources for others who
use user stories to case out design or implementation of
projects. Describe some of the pitfalls. Briefly cite and
describe one alternative to user stories. 1-2 pages}
