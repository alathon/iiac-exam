\setcounter{section}{0}
\setcounter{figure}{0}
\setcounter{table}{0}
\newpage
\textbf{\LARGE{{Part two: Analysis, discussion and reflection}}}\\

Part two of this report explores the question of \textit{why}
the design process occured as it did. In section~\ref{sec:participatory}
we consider what a design process is, in section~\ref{sec:overview}
we provide a general description of techniques used in the design process,
in section~\ref{sec:process} we describe how the authors realized an IT design
project using the \must{} method and in section~\ref{sec:detailed} we
look at a few of the key choices and techniques that were employed while working
on the design report.

The reader is expected to be familiar with details about \gomonkey{} 
from ``Part one: Design Project''.

\section{Participatory design in IT} \label{sec:participatory}
IT projects vary widely in size, shape and form. From the potentially small, painless 
setup of Google Apps for a company of 3 employees, to the large-scale migration of
Jyske Bank's IT infrastructure after 2 years of planning\cite{jyskebank}.
A large portion of the work done in this report is based upon the general
practices and techniques of participatory design as proposed by \cite{bodker2004participatory}.
B\"o dker et al.\ describe a search for techniques, concepts and ideologies which may apply
to the \textbf{design and planning} of IT projects on a general level. They
define an IT design project as
%%
\vspace{5mm}
\begin{framed}
\textbf{IT design project}: A project conducted at a company that reveals
goals, defines problems, and indicates solutions, with the aim of designing
sustainable uses of IT based on a specific problem within the company.
\end{framed}
\vspace{5mm}
%%
The \must{}\cite{bodker2004participatory} method consists of an
\begin{inparaenum}[1)]
\item initiation phase
\item in-line analysis phase
\item in-depth analysis phase and
\item innovation phase.
\end{inparaenum}
It seeks to provide a perspective on IT design where you view IT projects not
just through the lens of what a prospective client wishes done, but also take into
account the work environment, the organization as a whole, the interaction
between company and customers and subsequent changes and impacts that making
alterations to a work environment can result in. As a whole, participatory
design adopts an ethnographic approach to design, encouraging the designer to
get up close and personal.

These goals are supported by four guiding principles: the principle of a coherent 
vision, the principle of genuine user participation, the principle of firsthand
experience and the principle of anchoring visions. The four principles will not
be described in length here, instead some of the benefits are described below.

\begin{description}
    \item [The coherent vision] mandates that the existing IT systems, the work organization 
        and the qualifications of employees be taken into account. Amongst other things, 
        this helps avoid a common pitfall of a new IT system requiring training of employees 
        because they lack the skills to make use of it. 
    \item [Genuine user participation] is about using mutual learning activities
        to teach users about technical options, issues and solutions and to
        teach the designers about the work that the users are doing. There are
        two central goals that arise from this: increasing the chance that a
        finished system reflects \textbf{actual} work practices and
        requirements and increasing the chance that a finished system is used
        according to its original intentions.
    \item [Firsthand experience with work practices] realizes that being told
        about how something occurs is not the same as reading about it, nor is
        it the same as actually experiencing the practice first-hand. All three
        approaches are applicable and should occur in a design project, but
        there may be factors influencing a re-telling of work practices or a
        written description that only surface during first-hand experience such
        as an interview, an observation or a think-aloud experiment.
    \item [Anchoring visions] views the design process as a mutual learning
        process between all members of the project group, in order to gain a
        common understanding of the results produced by the design
        project\cite{bodker2004participatory}. This serves both to eliminate
        misunderstandings and to eliminate uncertainty that employees or management
        may have about proposed changes, instead transforming uncertainties into
        a positive work atmosphere aimed at improvement and innovation. The
        importance of this factor is underlined by other research such as
        \cite{standish20012}, who list the emotional maturity and ecosystem of
        the project as one of the most important components of successful IT
        projects.
\end{description}

The purpose of an IT design project is \textbf{not} to end up with a
finished implementation or actual, implemented alterations to company work
practice, but to provide a report that allows a company to make an
\textbf{informed choice} about a specific strategy or implementation plan to
adopt. Such a choice is an exercise in risk management. In 1995, The Standish
Group found the top four reasons for project success were user involvement,
executive management support, clear statement of requirements and proper
planning\cite{standish1995chaos}. Conversely, the top four reasons for failure
were found to be incomplete requirements, lack of user involvement, lack of
resources and unrealistic expectations respectively\cite{standish1995chaos}.
They also found that only about \textbf{19\%} of IT projects succeeded on-budget
and on-time. These kinds of failures are often accompanied by a very real, costly risk. 
In 2005, Bronte-Steward proposed that approximately \$500 billion was wasted
worldwide each year on IT purchases that failed to reach their
objectives\cite{bronte2005developing}. In 2006, the number of successful
projects as measured by The Standish Group\cite{standish2012} had increased to
\textbf{37\%}, attributed in part to the same four factors of success as they had
proposed in 1995. 

There are different ways to approach solving the reasons for
failure proposed by The Standish Group as well as how to reach the elements of
success. One approach is to focus on the \textit{implementation} stage of
software projects and improve the development process itself, e.g.\ the
innovation of methods such as agile development. Another is to focus on the
\textit{planning} stage of software projects, which is what \must{} and this
report concern themselves with.

Readers are encouraged to read \cite{bodker2004participatory} for a more
wholesome description of how the \must{} method works.

\section{Overview of techniques employed} \label{sec:overview}
The \must{} method describes 16 specific techniques that can be used, depending
on the phase of the project. Each technique serves a specific purpose, in that it
has pros and cons as far as information gathering, information organization or discovery
is concerned. Some techniques apply to specific phases of the \must{} method, and others
are relevant throughout the lifetime of the design project.

The techniques listed below were used in one or more of the design phases.
\begin{description}
    \item [Baseline Planning]: the division of work into a series of phases,
        each of which is separated by a baseline (e.g.\ a milestone), that has well-defined
        requirements or goals. An initial set of baselines is produced during the initiation
        phase, which is then re-evaluated and detailed further as steps are completed and new
        baselines reached. The technique dates back to \cite{andersen1990professional}, and is
        analogous to planning a project in terms of milestones, goals and actionable items. This
        particular technique is a recurring task of the entire design project.

    \item [In-situ interview]: interviews are one of the most frequently used techniques in IT 
        design, particularly qualitative interviews\cite{bodker2004participatory}. An in-situ
        interview is one that occurs \textit{at} the workplace, as opposed to occurring at a
        remote location away from the work environment. The purpose of such an interview is often
        to clarify current work practice, tasks and function. Interviews are often transcribed as-is
        or summarized into a list of points.

    \item [Observation]: an observation provides first-hand experience of concrete work practices,
        either that of the existing workflow or of a prototype or newly implemented system. Observation
        is a potentially time-consuming task, and care must be taken about how and what to record or transcribe
        as the immediate result of the observation. Observations can thus result in either video, audio or text,
        depending on what makes sense.

    \item [Diagnostic Mapping]: a diagnostic map focuses on problem areas, causes of those problems, their consequences
        and subsequently one or more proposals for solutions. This technique applies primarily to the in-depth analysis phase,
        to relate problems and solutions in an easy-to-read fashion. Each row in a diagnostic map potentially represents a story 
        that can be read using narrative such as, '\textbf{X} is a problem, caused by \textbf{A},\textbf{B} and \textbf{C}, having 
        the consequence of \textbf{F}. So we should solve \textbf{X} by doing \textbf{Y}'. \todo{MGG: It's clear to me that a diagnostic
        map lacks the prioritization of a risk assessment. I think this is a problem with a diagnostic map!}

    \item [Market study]: a market study is not listed as an explicit \textit{technique} by \cite{bodker2004participatory}, but often
        occurs in relation to IT projects to scope out prospective possibilities for target systems. Sometimes designing a custom system
        is the correct solution, while other times seeking out an existing solution and either using it as-is or tailoring it to use is
        the more sensible approach. A market study attempts to provide an unbiased evaluation of a number of different products or systems,
        such that they may later be compared more directly with the prioritized goals and aims of, in this case, a design report.

    \item [Think-aloud experiment]: as the name implies, a think-aloud experiment is one in which a subject performs
        a series of tasks and expresses as much of their decision process as they can, whilst completing tasks. The
        purpose is often to get inside the mind of the subject, by gaining insight into the \textit{reasons} for performing
        actions or for continuing down a certain path of action. It also provides first-hand insight into current work practices.

    \item [Document analysis]: a lot of potentially interesting and relevant information is often \textit{hidden away} in documents
        that occur as part of a work practice. Permission slips and responsibility waivers, booking forms and so on. Document analysis
        aims to extract pertinent information from one or more documents that are part of the work practice. There are a large number of
        different kinds of documents associated with a company, and different documents help provide different information. Annual reports
        and organizational diagrams may help tell the design group about the company as a whole, while specific documents used as part of
        a task may help shed light on the way that task occurs and the flow of information either inside the company or between customer
        and company.

    \item [Developing scenarios]: this technique originates with \cite{clausen1993narratives} and revolves around creating
        prose-style representations exemplifying a certain work practice, under future use of the system. This both anchors employees
        and helps develop a coherent vision of the future for all participants of the design group. A more common name for this technique
        is possibly the development of \textit{user stories} or \textit{usecases}, which embody a similar if not the same intent.

    \item [Experimenting with prototypes]: a mockup-based prototype can provide
        valuable information, if for example, it is applied to a number of user
        stories that an employee tries to get through. This technique helps
        anchor employees, generate ideas for usage and correct mistakes about
        workflow and organization. \cite{bodker2004participatory} differentiates
        between horizontal (a non-functional prototype) and vertical prototypes,
        and under that definition a mockup-based prototype is a
        \textbf{horizontal prototype}.
\end{description}

\section{From beginning to end: Our process of design} \label{sec:process}
\todo[inline]{Describe what we did from the very beginning
til the end. 2-3 pages}

\section{A detailed look at key choices and techniques} \label{sec:detailed}
\subsection{Combining the in-depth and innovation phases}
\todo[inline]{Describe how we combined the in-depth and
innovation phases, and how the literature also supports
this in certain cases. 2-3 pages}

\subsection{Ethnographically inspired analysis}
\todo[inline]{Describe how to combine techniques to provide
an ethnographically inspired analysis. It is not good enough
to know \textit{what} they do, we must also know \textit{why}.
1-2 pages}

\subsection{Observation as a technique}
\todo[inline]{Describe how observation works as a technique.
Describe how others have used observation to positive effect.
Describe how we have used observation, and what some of the
pitfalls are. Briefly cite and describe one alternative to
observation. 1-2 pages}

\subsection{User stories as a technique}
\todo[inline]{Describe how user stories work as a technique
to provide overview and structure workflow. Describe how we
performed this and provide relevant sources for others who
use user stories to case out design or implementation of
projects. Describe some of the pitfalls. Briefly cite and
describe one alternative to user stories. 1-2 pages}
