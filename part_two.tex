\setcounter{section}{0}
\setcounter{figure}{0}
\setcounter{table}{0}
\newpage
\textbf{\LARGE{{Part two: Analysis, discussion and reflection}}}\\

\todo[inline]{Lead-in text for this part. Describe what the
section aims to provide. 1/2 to 1 page.}

Part two of this report explores the question of \textit{why}
the design process occured as it did. In section~\ref{sec:participatory}
we consider what a design process is, in section~\ref{sec:overview}
we provide a general description of techniques used in the design process,
in section~\ref{sec:process} we describe how the authors realized an IT design
project using the \must{} method and in section~\ref{sec:detailed} we
look at a few of the key choices and techniques that were employed while working
on the design report.

The reader is expected to be familiar with details about \gomonkey{} 
from ``Part one: Design Project''.

\section{Participatory design in IT} \label{sec:participatory}
\todo[inline]{Describe what participatory design is, describe the
phases of design proposed by the course book and what each phase
represents and why each phase is important.}

IT projects vary widely in size, shape and form. From the potentially small, painless 
setup of Google Apps for a company of 3 employees, to the large-scale migration of
Jyske Bank's IT infrastructure after 2 years of planning\cite{jyskebank}.
A large portion of the work done in this report is based upon the general
practices and techniques of participatory design as proposed by \cite{bodker2004participatory}.
B\"o dker et al.\ describe a search for techniques, concepts and ideologies which may apply
to the \textbf{design and planning} of IT projects on a general level. They
define an IT design project as
\begin{figure*}[h!]
    \centering
    \textbf{IT design project}: A project conducted at a company that reveals
    goals, defines problems, and indicates solutions, with the aim of designing
    sustainable uses of IT based on a specific problem within the company.
\end{figure*}

The \must{}\cite{bodker2004participatory} method consists of an
\begin{inparaenum}[1)]
\item initiation phase
\item in-line analysis phase
\item in-depth analysis phase and
\item innovation phase.
\end{inparaenum}
It seeks to provide a perspective on IT design where you view IT projects not
just through the lens of what a prospective client wishes done, but also take into
account the work environment, the organization as a whole, the interaction
between company and customers and subsequent changes and impacts that making
alterations to a work environment can result in. As a whole, participatory
design adopts an ethnographic approach to design, encouraging the designer to
get up close and personal.

These goals are supported by four guiding principles: the principle of a coherent 
vision, the principle of genuine user participation, the principle of firsthand
experience and the principle of anchoring visions. The four principles will not
be described in length here, instead some of the benefits are described below.

\begin{description}
    \item [The coherent vision] mandates that the existing IT systems, the work organization 
        and the qualifications of employees be taken into account. Amongst other things, 
        this helps avoid a common pitfall of a new IT system requiring training of employees 
        because they lack the skills to make use of it. 
    \item [Genuine user participation] is about using mutual learning activities
        to teach users about technical options, issues and solutions and to
        teach the designers about the work that the users are doing. There are
        two central goals that arise from this: increasing the chance that a
        finished system reflects \textbf{actual} work practices and
        requirements and increasing the chance that a finished system is used
        according to its original intentions.
    \item [Firsthand experience with work practices] realizes that being told
        about how something occurs is not the same as reading about it, nor is
        it the same as actually experiencing the practice first-hand. All three
        approaches are applicable and should occur in a design project, but
        there may be factors influencing a re-telling of work practices or a
        written description that only surface during first-hand experience such
        as an interview, an observation or a think-aloud experiment.
    \item [Anchoring visions] views the design process as a mutual learning
        process between all members of the project group, in order to gain a
        common understanding of the results produced by the design
        project\cite{bodker2004participatory}. This serves both to eliminate
        misunderstandings and to eliminate uncertainty that employees or management
        may have about proposed changes, instead transforming uncertainties into
        a positive work atmosphere aimed at improvement and innovation. The
        importance of this factor is underlined by other research such as
        \cite{standish20012}, who list the emotional maturity and ecosystem of
        the project as one of the most important components of successful IT
        projects.
\end{description}

The purpose of an IT design project is \textbf{not} to end up with a
finished implementation or actual, implemented alterations to company work
practice, but to provide a report that allows a company to make an
\textbf{informed choice} about a specific strategy or implementation plan to
adopt. Such a choice is an exercise in risk management. In 1995, The Standish
Group found the top four reasons for project success were user involvement,
executive management support, clear statement of requirements and proper
planning\cite{standish1995chaos}. Conversely, the top four reasons for failure
were found to be incomplete requirements, lack of user involvement, lack of
resources and unrealistic expectations respectively\cite{standish1995chaos}.
These kinds of failures are often accompanied by a very real, costly risk. In
2005 Bronte-Steward proposed that approximately \$500 billion was wasted
worldwide each year on IT purchases that failed to reach their
objectives\cite{bronte2005developing}. In 2006, the number of successful
projects as measured by The Standish Group\cite{standish2012} had increased to
37\%, attributed primarily to the same four factors of success as they had
proposed in 1995. There are different ways to approach solving the reasons for
failure proposed by The Standish Group as well as how to reach the elements of
success. One common approach is to focus on the \textit{implementation} stage of
software projects and improve the development process itself, seen in the form
of e.g.\ agile development. Another is to focus on the \textit{planning} stage
of software projects, which is what \must{} and this report concern themselves with.

Readers are encouraged to read \cite{bodker2004participatory} for a more
wholesome description of how \must{} works.

\section{Overview of techniques employed} \label{sec:overview}
\todo[inline]{Describe a little about design techniques in general, cite
the coursebook and relevant articles. 3-6 sentences}

The \must{} method describes 16 specific techniques which are organized into
phases, principles, knowledge areas and the tools used to represent them. Many
of the techniques 
The techniques listed below were used in one or more of the design phases.
\todo[inline]{Below: Describe every technique used and how it was used, cite 
relevant articles for other uses or longer description}
\begin{description}
    \item [Baseline Planning]
    \item [Interview]
    \item [Observation]
    \item [Diagnostic Mapping]
    \item [Market study]
    \item [Think-aloud experiment]
    \item [Document analysis]
    \item [Developing scenarios]
    \item [Experimenting with prototypes]
    \item [Workshop]
    \item [Future workshop]
\end{description}

\section{From beginning to end: Our process of design} \label{sec:process}
\todo[inline]{Describe what we did from the very beginning
til the end. 2-3 pages}

\section{A detailed look at key choices and techniques} \label{sec:detailed}
\subsection{Combining the in-depth and innovation phases}
\todo[inline]{Describe how we combined the in-depth and
innovation phases, and how the literature also supports
this in certain cases. 2-3 pages}

\subsection{Ethnographically inspired analysis}
\todo[inline]{Describe how to combine techniques to provide
an ethnographically inspired analysis. It is not good enough
to know \textit{what} they do, we must also know \textit{why}.
1-2 pages}

\subsection{Observation as a technique}
\todo[inline]{Describe how observation works as a technique.
Describe how others have used observation to positive effect.
Describe how we have used observation, and what some of the
pitfalls are. Briefly cite and describe one alternative to
observation. 1-2 pages}

\subsection{User stories as a technique}
\todo[inline]{Describe how user stories work as a technique
to provide overview and structure workflow. Describe how we
performed this and provide relevant sources for others who
use user stories to case out design or implementation of
projects. Describe some of the pitfalls. Briefly cite and
describe one alternative to user stories. 1-2 pages}
