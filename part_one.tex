\textbf{\LARGE{Part one: Design Project}}
\todo[inline]{Part one is a design project report as described
    in the coursebook. The audience for this is the companies
    management. See e.g.\ page 186ff.
    Length between 25-30pp}

\section{Objective}

\subsection{Objective of the design project}
\todo[inline]{Quick recap of the requirements from the manager, and what goals 
    the system is designed to fulfill}

\subsection{Results from the in-line analysis}
\todo[inline]{What areas of the business exists, and what problems do they 
    entail}

\subsection{Results from the in-depth analysis}
\todo[inline]{Observations, standard systems, work practices}

\subsubsection{Standard systems}
\todo[inline]{Describe standard systems tested and why they do not meet our 
requirements.}

We compiled a list of possible standard systems and isolated two very possible
candidates. After deep testing sessions of both systems, we realized that none
of them fulfill the defined requirements in the design project.

\begin{description}
\item[SuperSaaS]
We decided not to use SuperSaaS, since the booking doesn’t support the required 
fields for the booking. A 'form' can be attached to a booking after submitting 
initial information (intial information being the date and time of booking, 
name, phone number and email address), and this 
form can include extra information about amount of people in each age group, the
type of customer and food. This information can only be seen with two clicks 
from the calendar overview, thus making it impossible to get a good overview of 
how 'available' a time slot is.

Furthermore, it is not possible to calculate a price based on the custom fields 
in the form, and as such, the system cannot ask for a deposit so we will still 
need another payment processor.

\item[onlinebooq.dk]
The other promising booking system was onlinebooq.dk. They support selling 
'services', but unfortunately they only support selling one of each serving, i
making it useless in this case, since the price is calculated from the amount of
people utilizing each service. It does support paypal and quickpay, but this 
doesn't matter if we cannot calculate the price correctly.
\end{description}


\section{The coherent vision}
\subsection{Technology and systems}
\subsubsection{IT systems and platform}
\todo[inline]{What do we need for running the system. Servers, software, etc}

\subsubsection{Functions of the system}
\todo[inline]{Describe what the system does}

\subsubsection{User interaction}
\todo[inline]{Pictures of mockup and descriptions of how it works}

\subsection{Work organization}
\todo[inline]{Do we change how the work organization is, someone with new
responsibilities? Can someone potentially do some of the managers work?}

\subsection{Qualification needs}
\todo[inline]{Does the system require any education? The manager will probably
need to tell about the new system to the instructors, and the manager will have
to be 'taught' how to use the new system.}
	
\subsection{Advantages and disadvantages}
\subsubsection{Business- and IT strategies}
\todo[inline]{How the system will help reach the goals described in the
    business- and IT strategies.}

\subsubsection{Groups of staff and their relations}
\todo[inline]{Advantages and disadvantages from an employees and the managers 
    perspective, and how the it system affects their communication and 
    relation.}

\subsubsection{Customers}
\todo[inline]{Advantages and disadvantages from a customers perspective.}


\section{Finances?}
\todo[inline]{Does this even make sense for the company?}


\section{Implementation strategy and plan}
\subsection{Technical implementation}
\subsection{Organizational implementation}

\section{Recommendations and priorities}

\newpage
